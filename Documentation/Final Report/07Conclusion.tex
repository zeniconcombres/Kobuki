\section{Conclusion}
The Kobuki project definitely challenged the team to think outside the box when exploring solutions and we did that by trying as many options as we could until we narrowed down to a most efficient implementation. Executing a final state machine in C using function pointers we hope was something that stretched our comfort zones, which it did, and though our Kobuki did not succeed in the final obstacle course, we hope that we presented an innovative method. There were a lot of considerations in the project with many interdependent factors. A main learning was to encourage ``anticlockwise thinking" when testing to cover all grounds for any situation (favourable or unfavourable) that may occur. Often when ``embedded" in development it is hard to see a different route in development. We encourage future projects to take some time off from the ``cyber" side and share their work in the ``physical" world with perhaps funky grandmas (to test your understanding when explaining) or younger cousins (to inspire them to become engineers).

%apparently we're supposed to come up with creative recommendations for futher works